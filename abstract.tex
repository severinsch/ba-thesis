\addcontentsline{toc}{chapter}{Abstract}
\vspace*{2cm}

\begin{center}
{\usekomafont{section} \textbf Abstract}
\end{center}
\noindent


As more and more aspects of our daily lives move online, the security of online data becomes increasingly important. SQL injections are among the most common threats to online security, and detecting potential vulnerabilities in source code is crucial to prevent such attacks. To successfully detect such vulnerabilities in source code, static analysis tools need information about the strings from which the queries are built.

In this thesis, we adapt an approach to extract formal grammars that describe strings from a code property graph. We combine this approach with other techniques to obtain a human readable regular expression, matching all possible values of the analyzed string.
We provide a proof-of-concept implementation and show that it is able to accurately describe complex strings. We think, that with further development, the information our approach provides can be used to successfully detect SQL injection vulnerabilities.



\begin{comment}	

In the last couple of years, I have supervized numerous bachelor's and master's
thesis and various seminars. This led to a broad observation of typical
questions and issues the students faced when writing their thesis or papers.
Surprisingly, they are always quite similar. This template aims to give advise
to future sstudents in order to answer the most frequent questions and avoid the
most common mistakes. It provides the TUM template which has already been
accepted many times, shows the most basic outline and some tips on the contents
of each chapter. It further contains some tips on the style of scientific works.
An evaluation on a small set of students showed that this guideline can assist
in making progress faster. However, we found that we have to keep improving the
tips to achieve better results.

Your abstract goes here. The typical structure is:
\begin{itemize}
\item Broad description of the current state
\item Gap in the current state
\item Your contribution
\end{itemize}
	content...
\end{comment}