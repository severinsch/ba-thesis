\chapter{Introduction}
\label{chapter:introduction}

\begin{comment}
	Your introduction goes here
	\begin{itemize}
		\item Generic description of the broad field of research
		\item Current state of research
		\item What's the gap that you're trying to fill?
		\item Short motivation
		\item Summary of the most important results
		\item Your contribution
		\item Structure of the thesis
	\end{itemize}
	
	1-2.5 pages
	
	This text is not too detailed. Start quite high-level, then narrow down until
	you reach your topic. After the introduction, the reader must want to read the
	rest of your thesis and understand the relevance. However, it doesn't have to
	be super technical.
\end{comment}

We first provide the motivation for our work in Section \ref{sec:intro:motivation}, then describe our contribution to the field in Section \ref{sec:intro:contribution}, and further establish the outline for the rest of the thesis in Section \ref{sec:intro:outline}.

\section{Motivation}\label{sec:intro:motivation}

The increasing reliance on software applications in various aspects of modern life has led to a growing concern for the security of these applications. Among the many security threats that can affect software, injection vulnerabilities are among the most dangerous and prevalent. The Open Web Application Security Project (OWASP) consistently lists injection attacks, which include SQL injections, LDAP injections, and command injections, as one of the top ten web application security risks \cite{owasp}.

Injection vulnerabilities occur when an attacker is able to insert malicious code or strings into an application, often through input fields that accept user input, such as search boxes or login forms. This can result in the attacker gaining unauthorized access to sensitive data, executing arbitrary code, or even taking control of the entire system.

To assist developers and security researchers in spotting injection vulnerabilities in code, a variety of tools and techniques, including static analysis tools for string values exist. These tools analyze the source code of an application to identify potential vulnerabilities, including injection vulnerabilities. Static analysis tools are particularly useful because they can detect vulnerabilities that may not be apparent during testing or manual code review. Making developers aware of such issues during development enables them to fix the vulnerability early. One such static analysis tool is Codyze\footnote{https://www.codyze.io}, which uses the implementation we extend in this thesis for its analyses.

In order to detect injection vulnerabilities, these tools may try to analyze the possible values a string that is passed as a query to e.g. a database can take on.
From these inferred properties, a tool can then assess whether the analyzed program contains any potential injection vulnerabilities and alert the programmer.

Since the analyzed strings often contain unknown user inputs, enumerating all possible values is not feasible. 
Therefore, we use regular expressions to describe the properties of all possible values.
For example, consider a string variable, that is used as an SQL query and is determined to be described by the regular expression \Verb@DELETE \* FROM myTable WHERE id='.*'@.
This information can be used to issue a warning during a static analysis, because the analyzed program allows for an arbitrary unchecked string to be inserted into the SQL query, which poses a severe security vulnerability.

\section{Contribution}\label{sec:intro:contribution}

In this thesis, we extend a \acf{cpg} implementation \cite{cpg} to increase its capabilities in analyzing string values. We adapt the theoretical approach by Christensen et al. \cite{brics}, which creates regular languages describing the values of a string, to the present \ac{cpg} implementation. Note that Christensen et al. work on a different representation of the analyzed code and use the results in a different way. For example, they use a different definition of the \ac{dfg} and a novel data structure representing the complete analyzed code, from which they extract \acp{dfa}. We only analyze parts of the code that are required for the specific query. Therefore, using their approach requires adaptation of the used techniques and solving of new problems like the handling of string operations for our use case. We also want to provide the information to an analyst in a human-readable format, in our case in the form of regular expressions.
To achieve this, we first combine the mentioned approach with the Nederhof algorithm \cite{nederhof} to transform the obtained results to automata. Further, we use state elimination in combination with a heuristic by Delgado and Morais \cite{delgado} to convert them to regular expressions for users. We also provide a proof of concept implementation covering a subset of the Java standard library.

\section{Outline}\label{sec:intro:outline}

We first describe the problems and posed research questions in Chapter \ref{chapter:ProblemDescription}.
After providing some theoretical background in Chapter \ref{chapter:Background}, we then present selected related work in Chapter \ref{chapter:RelatedWork} and describe the different steps of our approach in Chapter \ref{chapter:Approach}.
In Chapter \ref{chapter:Evaluation}, we evaluate the results and benchmark our implementation. There, we also highlight some limitations of our approach and evaluation and include ideas for future continuation and improvement of the presented design before concluding the outcome of this thesis in Chapter \ref{chapter:Conclusion}.
