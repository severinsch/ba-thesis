\chapter{Problem Description}
\label{chapter:ProblemDescription}

We extend a \ac{cpg} implementation by the Fraunhofer AISEC research institute \cite{cpg}. It currently has no means of providing information about the structure and contents of string variables that go beyond propagating literals if they are not changed.

Describing such strings is not trivial, as often at least part of a given string stems from an unknown source, for example runtime user input.
We solve this issue by first describing a given string with a formal grammar, which conservatively approximates the values the string can take.
This means that for a string $s$, the language generated by the grammar we obtain to describe $s$ always contains all possible values $s$ can have.
We want to use regular expressions as a final representation because they are human-readable and therefore allow users to manually evaluate our results for a security analysis. Regular expressions also allow automated tools to match given queries. 

The obtained grammars generate context free languages, which are a superset of the regular languages accepted by regular expressions. Therefore we can't directly convert the obtained grammars to regular expressions without any loss of information, but rather need to approximate them with grammars that generate regular languages first.

This approximation poses the challenge of deciding, which information to retain and which parts to change. We also need to account for the effects of operations like a \lstinline|replace| function on the analyzed strings. Additionally, it is necessary that our approximation stays conservative.

Furthermore, since the grammars our approximation creates are not textbook regular grammars, but rather strongly regular grammars, we need to use algorithms suited to this type of grammar for the conversion.

Because we use automata as an intermediary step in the conversion from grammar to regular expression, we use the state elimination algorithm \cite{brzozowksi_mccluskey} to convert a \ac{nfa} to a regular expression. As we want our results to be human-readable, we need to reduce the length of the resulting regular expression by optimizing the state elimination algorithm.
\clearpage
To summarize, this section poses the following research questions.

\begin{enumerate}[noitemsep]
	\item What approaches for obtaining information about string values exist?
	\item Can such an approach be adapted to the characteristics of the given \ac{cpg} implementation?
	\item How can we approximate the obtained grammar?
	\item How can we transform the approximated grammar into a human-readable format?
	\item How can we resolve the effect of operations on strings during this transformation?
	\item How can we improve the quality of our result and the performance of our approach?
\end{enumerate}

\begin{comment}
	
The introduction is a bit like a teaser. Here, you dig more into details, also
technical ones. After this chapter, the reader must understand why you do this
work, why it's important, what makes it difficult and what you want to achieve.

\begin{itemize}
\item What's the problem that you're trying to solve?
\item What is your goal?
\item What is/are the research question(s)?
\item What are special problems?
\end{itemize}

Probably 1-3 pages

\end{comment}
