\chapter{Approach and Implementation}
\label{chapter:Approach}

The general approach for our implementation is adapted from the one described by Christensen et al. \cite{brics}. Conceptually, we first extract a \ac{cfg} from the \ac{dfg}. In multiple steps using different methods this grammar we then approximate this grammar into a regular grammar. From this regular grammar we can create a regular expression object to provide for further analysis.

\tikzstyle{grammar} = [
	rectangle, 
	draw, 
	fill=gray!20, 
	text width=5em, 
	text centered, 
	rounded corners, 
	minimum size=2cm,
	thick
	]
\tikzstyle{graph} = [
	circle, 
	draw, 
	fill=orange!20, 
	text width=5em, 
	text centered,
	minimum size=2cm,
	thick
	]
\tikzstyle{result} = [
	rectangle, 
	draw, 
	fill=green!20, 
	text width=5em, 
	text centered, 
	minimum size=2cm,
	thick
	]
\tikzstyle{line} = [draw, -latex', very thick]
\begin{figure}[H]
	\centering
\begin{tikzpicture}[node distance=1.5cm, auto]
	    \node [graph] (DFG) {\ac{dfg}};
	    \node [grammar, right=of DFG] (CFG) {\ac{cfg}};
	    \node [grammar, right=of CFG, xshift=15mm] (REG) {regular grammar};
	    \node [result, right=of REG] (regex) {regular expression};
	    
	    \path [line] (DFG) -> (CFG);
	    \path [line] (CFG) -> node [midway] {approximation} (REG);
	    \path [line] (REG) -> (regex);
\end{tikzpicture}
\caption{The general approach for obtaining regular expressions}
\end{figure}
\label{fig:approach}

\section{Hotspot Collection}
- collect hotspots for grammar creation
- set of interesting locations
- start grammar creation from there

\section{Grammar Creation}
Our Grammar contains five types of productions:
\begin{itemize}
	\item \lstinline|UnitProduction|:  $X \rightarrow Y$
	\item \lstinline|ConcatProduction|: $X \rightarrow Y\ Z$
	\item \lstinline|TerminalProduction|: $X \rightarrow \text{<terminal>}$
	\item \lstinline|UnaryOpProduction|: $X \rightarrow op(Y)$
	\item \lstinline|BinaryOpProduction|: $X \rightarrow op(Y, Z)$
\end{itemize}

\begin{lstlisting}[label={lst:grammar_example}, caption={Example code},escapeinside={(*}{*)}, numbers=right, captionpos=b]
	String s(*\textcolor{red}{$^1$}*) = " foo";
	s(*\textcolor{red}{$^2$}*) = s(*\textcolor{red}{$^3$}*) + "bar";
	s(*\textcolor{red}{$^4$}*) = s(*\textcolor{red}{$^5$}*).trim();
\end{lstlisting}

\lstinline|UnitProduction|s mostly represent references between nodes where the underlying string is not changed. In \ref{lst:grammar_example} this would be the case for the reference from \lstinline|s|$^3$ to the variable declaration in line 1. 

\lstinline|ConcatProduction|s are created for \lstinline|BinaryOperator| nodes that represent string concatenation using the \lstinline|+| operator. For the example in \ref{lst:grammar_example} the nonterminal corresponding to the \lstinline|BinaryOperator| node for the \lstinline|+| in line 2 would have a \lstinline|ConcatProduction| with the right hand side nonterminals corresponding to the nodes for \lstinline|s|$^3$ and the string literal respectively.
% TODO implement append?

\lstinline|TerminalProduction|s point to a \lstinline|Terminal| that represents a fixed regular expression.
For example for the \lstinline|Literal| \ac{cpg} node representing the \lstinline|"bar"| string literal, the corresponding nonterminal has a \lstinline|TerminalProduction| where the \lstinline|Terminal| contains a regular expression that matches only the string "abc". \lstinline|TerminalProduction|s also occur at \ac{cpg} nodes without incoming \ac{dfg} edges where the value is not known. Those nodes could represent any string value and therefore the corresponding \lstinline|Terminal| contains the regular lanuage \lstinline|.*|, matching all strings.

\lstinline|UnaryOpProduction|s and \lstinline|BinaryOpProductions| represent function calls or other operators. The \ac{cpg} for \ref{grammar_example} contains a \lstinline|CallExpression| representing the function call of the library function \lstinline|trim|. We then create an \lstinline|Operation| object representing this operation and the \lstinline|UnaryOpProduction| $X \rightarrow trim(Y)$, where $X$ is the nonterminal corresponding to the node representing \lstinline|s|$^4$ and $Y$ to the one representing \lstinline|s|$^5$. The \lstinline|Operation| objects also contain information about possible arguments and implement the character set transformation and regular approximation needed for the approximation of the grammar described in \ref{approximation}. This language agnostic representation of string operation allows developers of the \ac{CPG} library to add support for functions and operators in other languages with different semantics compared to the corresponding Java functions, without needing to change the grammar approximation. For example for the Python expression \lstinline[language=Python]|"abc" * 5| the \lstinline|*| operator can be represented using a generic \lstinline|Repeat| \lstinline|Operation|.

To create the grammar for a given \ac{dfg} node, we start traversing the \ac{dfg} backwards, starting at the given node. For each visited node, we add a \lstinline|Nonterminal| to our grammar and the fitting productions.

Unlike Christensen et al. \cite{brics}, we do not consider the total \ac{dfg} when extracting the grammar. While they parse the whole graph into a data structure, to later extract automata for specific Nodes, we create the grammar starting from a single node and ignore all parts of the graph not connected via \ac{dfg} edges to this node.

Since often the majority of a large program is not relevant for a specific node, this reduces the amount of nodes we need to handle and the size of the resulting grammar, therefore leading to performance improvements.

Additionally, we can traverse the \ac{DFG} conditionally, stopping at nodes representing numbers. If the traversal reaches such a node, it uses a \lstinline|ValueEvaluator| to try, whether the value the node represents is known. In this case, we can add a \lstinline|TerminalProduction| with the \lstinline|Terminal| representing the value literal and otherwise, if the value is not known, the \lstinline|Terminal| contains a regular expression matching all numbers of the present type, e.g. \lstinline{"0|(-?[1-9][0-9]*)"} for integrals.
	

\section{Regular Approximation}\label{approximation}
\subsection{Character Set Approximation}
To use the Mohri-Nederhof approximation algorithm, we need to eliminate all cycles in our grammar that contain operation productions.
All nonterminals are assigned a character set, containing all characters that make up the words in the language of the corresponding nonterminal. Each operation defines a character set transformation - a function $T_{op} : 2^\Sigma \rightarrow 2^\Sigma$ - that approximates how the application of the given operation changes the character set. Here $\Sigma$ represents the set of all possible characters.
For example the character set transformation for a \lstinline|replace| operation, where a known char \lstinline|o| is replaced by a known char \lstinline|n| has the following character set transformation,

\begin{math}
	T_{replace[o, n]}(S) = 
	\begin{cases}
		(S \setminus \{o\}) \cup \{n\}, & \text{if } o \in S\\
		S, & \text{if } o \notin S
	\end{cases}
\end{math}

whereas for a \lstinline|replace| operation, where the newly inserted char is not known, $S$ is transformed to $\Sigma$ if the replaced char is contained in $S$.

These approximations, together with the terminals where the character set is known, for example a string literal, can be used in a fixed point computation to assign a character set $C(X)$ to each nonterminal $X$.

To break up the cycles containing operation productions, we replace one operation production $X \rightarrow op(Y)$ in each cycle with a production $X \rightarrow r$, where $r$ is the regular expression that matches the language $C(X)^*$.

To represent character sets easily, we have two different implementations, both conforming to a common \lstinline|CharSet| interface that requires functions like \lstinline|union : CharSet -> CharSet| and \lstinline|intersect : CharSet -> CharSet|.

The first, \lstinline|SetCharSet|, is mostly a simple wrapper around a \lstinline|Set<Char>| containing the characters.
The second, \lstinline|SigmaCharSet|, is used to easily represent sets like $\Sigma \setminus \{a, b, c\}$ by storing a \lstinline|Set<Char>| containing the characters not contained in the set, while all other characters are assumed to be members.

The behavior of the the set operations \lstinline|union| and \lstinline|intersect| can be described using the following set operations:

\noindent
\begin{alignat*}{3}
	& \text{\lstinline|SigmaCharSet union SigmaCharSet| } && \hat{=} (\Sigma \setminus A) \cup (\Sigma \setminus B) & &= \Sigma \setminus (A \cap B) \\
	& \text{\lstinline|SigmaCharSet union SetCharSet| } && \hat{=} (\Sigma \setminus A) \cup S & &= \Sigma \setminus (A \setminus S) \\
	& \text{\lstinline|SetCharSet union SetCharSet| } && \hat{=} & &\phantom{{}={}} S_1 \cup S_2 \\
	& \text{\lstinline|SigmaCharSet intersect SigmaCharSet| } && \hat{=} (\Sigma \setminus A) \cap (\Sigma \setminus B) & &= \Sigma \setminus (A \cup B) \\
	& \text{\lstinline|SigmaCharSet intersect SetCharSet| } && \hat{=} (\Sigma \setminus A) \cap S & &= S \setminus A \\
	& \text{\lstinline|SetCharSet intersect SetCharSet| } && \hat{=} & &\phantom{{}={}} S_1 \cap S_2
\end{alignat*}

This approach reduces the storage needed to represent the commonly occurring type of character sets, where only a few characters are removed from $\Sigma$. It also simplifies the creation of a regular expression from the character set, since the approach of using a character class containing all characters in the set produces very large character classes for sets with cardinality close to $|\Sigma|$. Using our approach, we can represent a \lstinline|SigmaCharSet| using negated character classes. Since most character sets either contain a comparatively small amount of given chars, or all chars except a few this reduces the average length of the resulting regular expressions. 
For example the \lstinline|SetCharSet| that represents the set $\{\text{'a', 'b', 'c'}\}$ gives us the regular expression \lstinline|[abc]*|, while the \lstinline|SigmaCharSet| representing $\Sigma \setminus \{\text{'0', '1', '2'}\}$ corresponds to \lstinline|[^012]*|.

		
\subsection{Mohri-Nederhof Approximation}

