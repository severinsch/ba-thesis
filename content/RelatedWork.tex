\chapter{Related Work}
\label{chapter:RelatedWork}

The challenge of statically obtaining information about the values of strings is not new and over the years there have been different approaches to it.

We follow the approach by Christensen et al. \cite{brics}. The authors construct a context free grammar from a flow graph, but instead of creating it on-demand, starting at the chosen hotspot node like we do, they consider the total flow graph for grammar creation.
As the total amount of potential hotspots can be far greater than the number of actual points of interest, this total grammar is potentially significantly larger than needed. This increased size reduces the performance of the subsequent steps.
Christensen et al. use the same approximation methods for obtaining regular languages from the generated context free grammars, but instead of directly transforming these grammars into automata like we do, they produce a novel formalism, the \ac{mlfa}.
This automaton allows the extraction of the respective automata for different hotspots. Due to the aforementioned on-demand generation of the grammar, we don't need the extraction capability for single hotspots that the \ac{mlfa} provides. Christensen et al. also don't transform the obtained automata to regular expressions, but rather provide query options using automata.
The authors provide a feature rich implementation\footnote{https://www.brics.dk/JSA/} of their approach and show that it efficiently produces useful results.


Tabuchi et al. \cite{regex_types} describe a type system for a minimal functional calculus, where strings have a regular expression as their type. Using typeinference and reconstruction algorithms, they assign such a type to each string variable. Properties about a variable can then be obtained from its type.
They show that their proposed type system can produce good results when applied to their minimal calculus. While we considered implementing this approach for the analysis, there are some problems, especially due to our different requirements and prerequisites.

To use the presented approach in practice an (efficient) algorithm for type checking and type reconstruction is needed. 
The given paper does not include those, but rather indicates several problems in constructing such algorithms for the given situation without losing some of the desired preciseness.
The authors mention that using standard type reconstruction by constraint solving for the proposed type system even is impossible due to limitations of regular languages.

Additionally this approach is tailored to the mentioned calculus and utilizes specific features like pattern matching, which would make adapting it to our use case more difficult.

The additional layer of abstraction introduced by the DFG used in the approach we chose eliminates this problem and makes adaption easier.


Wassermann and Su \cite{sqli_wassermann_su} present an approach comparable to ours, where they also characterize values of string variables using context free grammars. They specifically target SQL injection vulnerabilities by using the generated \acp{cfg} to check whether user input can change the syntactic structure of a query. While this approach is successful in detecting those vulnerabilities, our approach is more general and not focused on detecting one specific type of problem, but rather on providing general information for unspecified further use.