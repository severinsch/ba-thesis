\chapter{Background}
\label{chapter:Background}
\begin{comment}
What is the knowledge a undergrad student needs so that he/she can understand
your thesis? You can assume some familiarity with the very broad topic. E.g. if
you write a thesis in the area of software analysis, you do not have to explain
static/dynamic analysis as such (this is boring!). If you're a crypto guy, don't
explain AES in detail unless you try to break it in your thesis. If I stumble
across a word/term in your thesis and don't understand it, this is where I would
look it up (or on google).

Probably approx. 3-10 pages
\end{comment}

We first provide some background information and notation for formalisms used in the following chapters. This includes formal grammars in Section \ref{sec:background:grammars}, strongly regular grammars in Section \ref{sec:background:srg}, finite automata in Section \ref{sec:background:automata} and regular expressions in Section \ref{sec:background:regex}. In Section \ref{sec:background:cpg}, we then describe the code property graph we use.

\section{Formal Grammars}\label{sec:background:grammars}

A formal grammar consists of a set of nonterminal symbols $N$, an alphabet $\Sigma$ of terminal symbols, a set of production rules $(\Sigma \cup N)^*N(\Sigma \cup N)^* \rightarrow (\Sigma \cup N)^*$, also called just \enquote{productions} and a start nonterminal.

In the context of this thesis, $\Sigma$ is the set of characters making up the strings of the analyzed programming language. In some places, we use regular expressions as terminals, which is an extension to the usual definition.

To define grammars in examples, we use the notation $X \rightarrow Y$ to specify a production rule to transform some $X$ to some $Y$.
We use capital letters for nonterminals and lowercase letters for terminals. We also don't specify the start symbol explicitly, but rather the nonterminal on the left hand side of the first given production is considered to be the start symbol.

\Acfp{cfg} are grammars, where the left hand side of all productions consists of a single nonterminal. The set of production rules for \acp{cfg} therefore is defined as $N \rightarrow (\Sigma \cup N)^*$ compared to the general definition above.

Regular grammars are a subset of \acp{cfg}, where the right hand side of each production consists either of a single terminal $a \in \Sigma$ or of exactly one nonterminal and one terminal. Additionally, the nonterminals on the right hand side are either always the first symbol or always the last symbol on the right hand side. 

For example, a grammar containing both the productions $A \rightarrow aB$ and $B \rightarrow Ab$ with $A, B \in N \land a, b \in \Sigma$ therefore is not regular. Here, the nonterminal is the last symbol in the first production but the first symbol in the latter, which does not comply with the above rule.

\section{Strongly Regular Grammars}\label{sec:background:srg}

$\mathcal{R}$ is the equivalence relation defined on the set of nonterminals $N$ of some grammar:

\begin{align}
	A \mathcal{R} B \Leftrightarrow (\exists \alpha, \beta \in V^* : A \xrightarrow{*} \alpha B \beta) \land (\exists \alpha, \beta \in V^* : B \xrightarrow{*} \alpha A \beta) 
\end{align}

Here $V$ is $\Sigma \cup N$, so the set of all symbols, terminal and nonterminal. $\xrightarrow{*}$ is the reflexive and transitive closure of the production relation $\rightarrow$ defined by the set of productions in the grammar. $A \xrightarrow{*} \alpha B \beta$ means, that there exists a sequence of productions starting at the symbol $A$ to produce a set of symbols that contain $B$. Therefore $\mathcal{R}$ groups all nonterminals into disjoint equivalence classes, where each nonterminal in a class can be produced by each other nonterminal in the class. Those nonterminals are called mutually recursive.

A grammar is strongly regular if the production rules in each such equivalence class are either all right-linear or left-linear.

A production rule is right-linear if it is of the form $A \rightarrow w \alpha$, where $w$ is a sequence of terminal symbols and $\alpha$ is empty or a single nonterminal symbol. Left-linear productions are defined accordingly but the nonterminal is on the left side of the production result.
\Aclp{srg} are guaranteed to generate regular languages \cite{mohri_nederhof}.

\section{Automata}\label{sec:background:automata}

A \acf{dfa} consists of a set of states $Q$, an alphabet of input symbols $\Sigma$, a transition function $\delta: Q \times \Sigma \rightarrow Q$, an initial state $q_0 \in Q$ and a set $F \subseteq Q$ of accepting states.
For a \acf{nfa}, from a given state multiple states can be reached with the same input, so the transition function is  $\delta: Q \times \Sigma \rightarrow 2^Q$, where $2^Q$ denotes the power set of $Q$.

We represent automata as graphs, where each state is a node and the transition function is represented by edges labeled with elements of $\Sigma$. The start state is marked with an incoming arrow and the accepting states are marked with double circles.
An edge in this graph, henceforth also called transition, is denoted as $(q_1, a, q_2)$, where $q_1 \in Q$ is the origin state, $a \in \Sigma$ is the label and $q_2 \in Q$ is the target state of the edge.

\section{Regular Expressions}\label{sec:background:regex}

We use a regular expression syntax with metacharacters from the Java regular expression flavor, which can be seen in Table \ref{tab:regex}.

\begin{table}
	\centering
	\begin{tabular}{ccc}
			\toprule
		\thead{\textbf{Metacharacter}} & \thead{\textbf{Name}} & \thead{\textbf{Description}} \\
			\midrule
		\makecell{\Verb@*@} & \makecell{Kleene star} & \makecell{Matches the previous character\\zero or more times}\\
			\midrule
		\makecell{\Verb@.@} & \makecell{Wildcard} & \makecell{Matches any character}\\
			\midrule
		\makecell{\Verb@?@} & \makecell{Option} & \makecell{Matches the previous character\\zero or one time}\\
			\midrule
		\makecell{\Verb@|@} & \makecell{Choice} & \makecell{Matches either the previous or the\\following expression}\\
			\midrule
		\makecell{\Verb@[abc]@} & \makecell{Character class} & \makecell{Matches any of the contained characters\\(here \Verb@a@, \Verb@b@ and \Verb@c@)}\\
			\midrule
		\makecell{\Verb@[^abc]@} & \makecell{Negative character class} & \makecell{Matches any character not contained\\(here anything except \Verb@a@, \Verb@b@ and \Verb@c@)}\\
		\bottomrule
	\end{tabular}
	\caption{The regular expression metacharacters we use}
	\label{tab:regex}
\end{table}

Expressions are grouped using round brackets and meta characters escaped using single backslashes.
In a character class consecutive characters can be abbreviated using \Verb@-@, e.g. \Verb@[0-9]@ to match any digit.

\section{Code Property Graph}\label{sec:background:cpg}
A \acf{cpg} is a directed multi graph, where the nodes represent syntactic elements like simple expressions or function declarations and the edges represent the relations between those elements.
While the \ac{cpg} is a general concept, we focus on one specific implementation\footnote{https://github.com/Fraunhofer-AISEC/cpg}, that extracts a \ac{cpg} out of source code of a set of different programming languages.

There, nodes and edges have a list of key-value pairs called properties which contain general information for the element. For example, a node representing a statement in a source file contains the location of the underlying code and an edge representing evaluation order may contain whether the target statement is unreachable. The graph is initially created by language frontends, which create partially connected \acp{ast}, which are then enriched by additional information like the mentioned evaluation order and data flow information by multiple passes \cite{cpg}. A pass iterates the graph in some way to enhance it with, for example, additional edges or other information.
Users of the library can extend its functionality by adding additional passes.

While the \ac{cpg} contains many different types of edges, the most relevant edge type for this thesis is data flow edges, which represent the data flow between different expressions.

\begin{lstlisting}[label={lst:stringCreation}, caption={\ac{cpg} example code}, captionpos=b, numbers=right]
	String s = "xyz";
	System.out.println(s);
\end{lstlisting}

Consider the short code example in Listing \ref{lst:stringCreation}. Here, among others, the following nodes are part of the \ac{cpg}:

\begin{itemize}
	\item \lstinline|Literal|, representing the string literal \lstinline{"xyz"}
	\item \lstinline|VariableDeclaration|, representing the declaration and initialization of the variable \lstinline|s|
	\item \lstinline|DeclaredReferenceExpression|, representing the reference to the variable \lstinline|s| in line 2.
\end{itemize}

In this example, the data flows from the \lstinline|Literal| node to the \lstinline|VariableDeclaration| and from there to the \lstinline|DeclaredReferenceExpression|.

The nodes connected by those edges effectively form a subgraph of the \ac{cpg}, the \ac{dfg}, from which we then extract the information on string values.


